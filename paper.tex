\documentclass[12pt,a4paper]{article}

% Paquetes
\usepackage[utf8]{inputenc} % Codificación UTF-8
\usepackage[spanish]{babel} % Idioma español
\usepackage{amsmath, amssymb} % Matemáticas
\usepackage{graphicx} % Imágenes
\usepackage{geometry} % Configuración de márgenes
\usepackage{hyperref} % Hipervínculos
\geometry{margin=1in} % Márgenes de 1 pulgada

% Título y autor
\title{Título del Artículo}
\author{Nombre del Autor}
\date{\today}

\begin{document}

\maketitle

\begin{abstract}
Este es el resumen del artículo. Aquí se describe brevemente el contenido y los objetivos principales.
\end{abstract}

\section{Introducción}
Aquí comienza la introducción del artículo. Explica el contexto, los objetivos y la motivación del trabajo.

\section{Metodología}
Describe los métodos y procedimientos utilizados en el trabajo.

\section{Resultados}
Presenta los resultados obtenidos, incluyendo tablas, gráficos o figuras si es necesario.

\section{Discusión}
Analiza los resultados y compáralos con trabajos previos.

\section{Conclusión}
Resume los hallazgos principales y sugiere posibles trabajos futuros.

\section*{Referencias}
\bibliographystyle{plain}
\bibliography{bibliografia}

\end{document}